\documentclass[english,12pt]{aghthesis}

\usepackage[T1]{fontenc}
\usepackage[utf8]{inputenc}
\usepackage{upgreek}
\usepackage{polski}
\usepackage{url}

\author{Antoni Mleczko \\ Maciej Mionskowski}

\titlePL{Interaktywna wizualizacja instrukcji składania Origami z elementami symulacji fizyki papieru}
\titleEN{Interactive visualisation of Origami folding with elements of paper physics simulation}

\fieldofstudy{Informatyka}

\supervisor{dr inż. Witold Alda}

\date{\the\year}

\begin{document}

\maketitle

\section{\SectionTitleProjectVision}
\label{sec:cel-wizja}
\subsection{Problem characteristics}
Origami has been around for a long time. It originated in China and Japan and spread all around the world.\cite{wiki:history-of-origami} Origami is recognized as the art of paper folding. Recently it is becoming increasingly popular in a scientific context.
Mathematicians started recognizing material folding as a distinct branch known as computational origami. The field has seen a tremendous development in the past couple decades especially in the software space. Although dynamic, there are still many open problems \cite{mit-open-problems}. Beginners fold origami following step-by-step instructions. More advanced origamists will use a crease pattern to form a folded state. 
The process of origami creation consists of two stages: \textit{design} and \textit{folding}.
\begin{figure}
\end{figure}

\subsection{Motivation}

Even though origami seems to be a child's play, at times
people would get discouraged whilst following the origami instructions due to
the lack of details they expose.

We would like to provide a way for beginners (and advanced origamists alike) to visualise step-by-step the folding process of the origami pattern that they provide.
While there exist some programs aiding the process of design, there is no satisfactory solution
that would present the process of folding as it would be carried out manually.

We have evaluated existing solutions, and the one that resembles 
what we would like to achieve the most \cite{origami-simulator}, provides a way to load the crease pattern
and display the process of folding, however it does not allow to visualise the process step-by-step, it only allows 
to go from flat state to folded state immediately. It also bypasses some physical properties, such as the fact that the
paper should collide with itself.


\begin{figure}
\end{figure}

\subsection{Product vision}

Our main goal is to create a platform to visualise the origami model in 3D,
in every step of the folding process, while animating the transitions between the steps.

The inseparable component of the system would be a file format describing the folding steps neccesary to complete the origami.\\

\noindent \textbf{The users would be able to:}
\begin{itemize}
	\item load the folding instruction
	\item choose the step of the instruction that they want to visualise
	\item rotate the scene
    \item zoom in and out
	\item move around the scene
	\item pause the animation in any moment, and move back and forth
\end{itemize}


\subsection{Feasibility study}
\subsection{Threat assessment}

Playing with the intersection between reality and computer science has always been a challenging task. Here, we are trying to tackle a problem that has not been discussed previously.
Every year the field of computational origami sees a progress, so the mathematics behind it are not yet well developed. We foresee many challenges along the way, such as:
\begin{itemize}
	\item NP-hardness - some problems that we will face are in general proved to be NP-hard. Example of such a problem can be computing layer ordering based on a crease pattern. We will have to overcome them either using approximate methods, or coming up with solutions that will avoid them.

	\item physical properties of a paper - if we would like to support complex physical properties, there is a lot of features that would require a separate set of computations simulating paper physics, e.g.
		\begin{itemize}
			\item inflating
			\item curving 
			\item cutting
		\end{itemize}

	\item performance - web browser are stil not well optimized to carry out 3D computations and render 3D graphics in real time. The system will have to be highly optimized, in order to be usable. State of the art systems offload computational payload to GPU.

	\item mathematics - we don't have a lot of experience in writing complex simulations utilizing complicated mathematical formulas. Even computing a simple paper fold, requires physical computations such as computing strain on different parts of the paper.

	\item 3D graphics - we have some experience working with 3D, however only from the user perspective. We have little experience in creating 3D graphical software.

\end{itemize}

That all being said, we believe we will be able to undertake this problem and provide a solution to it.
While challenging it is also rewarding in terms of business value and as a unique product in the field.


\subsection{Dictionary}
\begin{description}
	\item[computational origami]
	\item[crease pattern]
	\item[flat state]
	\item[folded state]
	\item[mountain-valley assignment]
	\item[NP-hard]
\end{description}

\section{\SectionTitleScope}
\label{sec:zakres-funkcjonalnosci}
\emph{}  

\section{\SectionTitleRealizationAspects}
\label{sec:wybrane-aspekty-realizacji}
\emph{}  

\section{\SectionTitleWorkOrganization}
\label{sec:organizacja-pracy}
\emph{} 

\section{\SectionTitleResults}
\label{sec:wyniki-projektu}
\emph{}  

\bibliography{bibliography}

\end{document}
