\documentclass[polish,12pt]{aghthesis}

\usepackage[T1]{fontenc}
\usepackage[utf8]{inputenc}
\usepackage{upgreek}
\usepackage{polski}
\usepackage[polish]{babel}
\usepackage{url}

\author{Antoni Mleczko \\ Maciej Mionskowski}

\titlePL{Interaktywna wizualizacja instrukcji składania Origami z elementami symulacji fizyki papieru}
\titleEN{Interactive visualisation of Origami folding with elements of paper physics simulation}

\fieldofstudy{Informatyka}

\supervisor{dr inż. Witold Alda}

\date{\the\year}

\begin{document}

\maketitle

\section{\SectionTitleProjectVision}
\label{sec:cel-wizja}
\emph{Charakterystyka problemu, motywacja projektu (w tym przegl±d
  istniej±cych rozwi±zañ prowadz±ca do uzasadnienia celu prac),
  wizja produktu i analiza zagro¿eñ.}  

\section{\SectionTitleScope}
\label{sec:zakres-funkcjonalnosci}
\emph{Kontekst u¿ytkowania produktu (aktorzy, wspó³pracuj±ce systemy)
  oraz specyfikacja wymagañ funkcjonalnych i niefunkcjonalnych.}  

\section{\SectionTitleRealizationAspects}
\label{sec:wybrane-aspekty-realizacji}
\emph{Przyjête za³o¿enia, struktura i zasada dzia³ania systemu,
  wykorzystane rozwi±zania technologiczne wraz z uzasadnieniem
  ich wyboru, istotne mechanizmy i zastosowane algorytmy.} 

\section{\SectionTitleWorkOrganization}
\label{sec:organizacja-pracy}
\emph{Struktura zespo³u (role poszczególnych osób), krótki opis i
  uzasadnienie przyjêtej metodyki i/lub kolejno¶ci prac, planowane i
  zrealizowane etapy prac ze wskazaniem udzia³u poszczególnych
  cz³onków zespo³u, wykorzystane praktyki i narzêdzia w zarz±dzaniu
  projektem.}  

\section{\SectionTitleResults}
\label{sec:wyniki-projektu}
\emph{Wskazanie wyników projektu (co konkretnie uda³o siê uzyskaæ:
  oprogramowanie, dokumentacja, raporty z testów/wdro¿enia, itd.), prezentacja wyników
  i ocena ich u¿yteczno¶ci (jak zosta³o to zweryfikowane --- np.\ wnioski
  klienta/u¿ytkownika, zrealizowane testy wydajno¶ciowe, itd.),
  istniej±ce ograniczenia i propozycje dalszych prac.} 

% o ile to mo¿liwe proszê uzywaæ odwo³añ \cite w konkretnych miejscach a nie \nocite

\nocite{artykul2011,ksiazka2011,narzedzie2011,projekt2011}

\bibliography{bibliography}

\end{document}
