\subsection{Product characteristics}
% Charakterystyka projektu i sposób jego realizacji - macie tutaj napisać co to był za projekt (badawczy, rozwojowy) a wiec czy wymagania były jasne czy tak naprawdę powstawały w jego trakcie. Ważne by spróbować scharakteryzować przyjęty sposób realizacji a wiec przyjęty proces np. czy to był proces przyrostowy czy iteracyjny. Najczęściej Wasz proces jest oparty na prototypowaniu z odrzucaniem przez pierwszy semestr, a potem to typowy proces przyrostowo-iteracyjny w drugim semestrze. Spróbujcie napisać również z czego to wynika.

\subsection{Team}
% Osoby w projekcie - często projekt jest robiony dla kogoś a opiekun ma jakieś osoby pomocnicze. Wskażcie wszystkie osoby i ich role w projekcie. Bardzo ważny punkt z formalnego widzenia projektu, gdyż każda osoba musi zostać oceniona w sposób indywidualny.
% Zespól i podział obowiązków - musicie napisać co kto robił. Jeśli wszystko robiliście razem to spróbujcie wskazać chociaż 2-3 główne zadania, którymi każde z Was się zajmowało.

\subsection{Organization of work}
% Organizacja prac i wykorzystane narzędzia - używaliście facebooka, emaili czy spotykaliście się zawsze o określonej porze? Jak i kiedy dzieliliście się zadaniami: czy było to po każdym spotkaniu z klientem? Kiedy spotykaliście się z klientem? Jak często? Używaliście Jiry, confluence, trello? Koniecznie zróbcie screenshota! A może jakieś statystyki z githuba które jakoś podsumują Wasz projekt? Tutaj Wasze przemyślenia i komentarze mile widziane.
% Zastosowane techniki i praktyki - odnosi się do powyższego, ale możecie tutaj spróbować wskazać takie praktyki jak Pair Programming, TDD, Refactoring, Planning Game, User Stories, Team Velocity, Continuous Integration, Sprint Review Meeting, makiety GUI. Jak wyglądał proces testowania? Czy były robione jakieś prototypy? Jak były walidowane, odrzucane (te kwestie ew. do dokumentacji technicznej)?

\subsection{Project timeline}
% Przebieg prac (harmonogram, kalendarz): podział na poszczególne etapy, iteracje i jak one wyglądały i co było ich efektem.

\subsection{Deployments, tests, and experiments}
% Opis ewentualnych wdrożeń, testów, eksperymentów
